\chapter[~~~PRÉSENTATION]{~~~I -~Présentation de l'application}%
\label{refDev1}%

Ce premier chapitre présente l'application \nom\~et son contexte. Il détaille le sujet du projet, les choix réalisés par rapport à ce dernier et les extensions effectuées.

\section{Résumé du sujet}
\subsection{Introduction}

Ce projet est un projet de fin de seconde année de licence d'informatique au sein de l'université de Franche-comté, dans l'unité d'enseignement Programmation Objet Avancée.
Le projet a été réalisé en binôme, avec Monsieur CUINET Antoine et Monsieur AMIOTTE-SUCHET Tristan, des étudiants passionnés.

\subsection{Le projet}

Ce projet est réalisé en java (compatible java 11) en utilisant une programmation
objet mettant en œuvre les concepts d’héritage, de classe abstraite, d’interface, de généricité et de polymorphisme.
\bigskip

Ce projet est un exercice de programmation consistant à faire évoluer, automatiquement et au hasard, des personnages sur une grille où sont disposées des tours. 

Les personnages se déplacent d'une case à la fois en suivant une direction horizontale, verticale ou diagonale quand ils sont au sol. 
Ils rebondissent quand ils rencontrent un bord. 

Quand ils arrivent sur une case contenant une tour, ils entrent dedans se mettent à se déplacer en hauteur: ils peuvent monter les étages un à un jusqu'à arriver sur le toit. S'ils y parviennent ils deviennent propriétaires de la tour. 
Depuis le toit ils peuvent soit redescendre les étages un à un, soit sauter
sur le toit d'une autre tour à condition d'en être propriétaire.

\section{Choix par rapport au sujet}

Par rapport au sujet, nous avons réalisé tout ce qui était demandé, à savoir:

\begin{itemize}
    \item La rencontre avec un autre personnage.
    \item La rencontre avec un bord.
    \item La rencontre avec une tour:
    \begin{itemize}
        \item L'entrée dans la tour depuis le sol.
        \item La sortie de la tour par le sol.
        \item La sortie de la tour par le toit.
    \end{itemize}
\end{itemize}
\bigskip

Nous avons également pris soin de réspecter toutes les règles obligatoires d'implantation du sujet, sur les occupants, les positions, les directions, les redirection à la demande ainsi que sur les interfaces à implanter.

\section{Extensions réalisées}

Le sujet de base est un jeu de stratégie où des personnages évoluent de manière aléatoire sur une grille.

Nous avons réalisé plusieurs extensions par rapport au sujet de base, à savoir:

\begin{itemize}
    \item Des menus:
    \begin{itemize}
        \item Un menu de lancement du jeu, afin de le présenter.
        \item Un menu indiquant la fin du jeu et affichant les scores des différents joueurs.
    \end{itemize} 

    \item Des affichages:
    \begin{itemize}
        \item Un affichage des tours possédées par les joueurs en temps réel.
        \item Un affichage complet en couleur, avec les bord en rouge, les joueurs ayant tous une couelurs différentes (jusqu'a 6 couleurs différentes pour les joueurs) et les tours possédées ont la couleur du joueur qui la possède.
    \end{itemize} 

    \item Des entitées:
    \begin{itemize}
        \item Des \textit{Snowmans}, qui sont des bonhommes de neige qui empêchent le joueur qui le touche de se déplacer pendant un certain nombre de tours (6 par défaut), à la fin de ce nombre de tours, le bonhomme disparait et laisse le joueur continuer son chemin.
        \item Une tempête de météorites (\textit{MeteorRain}), qui envoie tout au long de la partie des météorites sur la grille (de façon aléatoire et avec une certaine probabilité (0,7 par défaut)). Une seule météorite maximum peut tombée sur la grille par manche. Lorsqu'elle tombe sur une case vide, on peut voir l'explosion mais rien ne se produit. Lorsqu'elle tombe sur une case occupé par un joueur, celui-ci est tué et il perd toutes ses tours. Une météorite ne peut pas tomber sur une tour, ni sur un bonhomme de neige.
    \end{itemize}

    \item Autres extensions:
    \begin{itemize}
        \item Une boucle de jeu permettant de relancer une nouvelle partie à la fin de la partie précédente, si le joueur le souhaite, ou de quitter le jeu.
        \item Un nombre de manches maximum à jouer, afin de déterminer le vainqueur avant que toutes les tours ne soient prises (ce nombre est réglé sur 50 par défaut mais peut être modifié dans la classe \textit{Main}), si toutes les tours sont prises avant la fin du nombre maximum de manche, la partie s'arrête.
    \end{itemize}
\end{itemize}
\bigskip

Le nombre de personnages, de tours, le temps d'une manche (en millisecondes), le nombres de manches, la taille de la grille\dots peuvent être modifiés dans la classe \textit{Main} (un commentaire indique le lieu de modification).
\bigskip

Nous avons donc implémenté une version avec de nombreuses fonctionnalités supplémentaires afin de rendre le jeu plus intuitif et plus complet.