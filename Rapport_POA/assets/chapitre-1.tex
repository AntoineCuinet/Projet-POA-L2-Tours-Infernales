\chapter[~~~PRÉSENTATION DE L'APPLICATION]{~~~I -~Présentation de l'application}%
\label{refDev1}%

Ce premier chapitre présente \dots

\section{Résumé du sujet}

\section{Choix par rapport au sujet}

\section{Extensions réalisées}


%%% 2 logos cote à cote
% \begin{figure}[h]
%     \begin{center}
%     \begin{minipage}[c]{.46\linewidth}
%     \centering \includegraphics[width=4cm]{assets/pictures/logos/logo_dklik.png}
%     \end{minipage}
%     \hfill
%     \begin{minipage}[c]{.46\linewidth}
%     \centering \includegraphics[width=4cm]{assets/pictures/logos/logo_amenothes.jpg}
%     \end{minipage}
%     \caption[\emph{Logos des agences}]{Logos des agences \textsc{D-Klik interactiv} et \textsc{Amenothès}.}%
%     \label{logoagences}%
%     \end{center}
% \end{figure}
% \bigskip


% \begin{itemize}
%     \item Le \textbf{pôle Référencement et Webmarketing}, qui a pour but de concevoir des stratégies de marketing, des campagnes influenceurs, etc et s'occupe du référencement naturel (SEO) et payant (SEA) (voir GLOSSAIRE); 
%     \item Le \textbf{pôle Web}, s'occupant de la création de sites web e-commerce dans le domaine de l'automobile, du sport et autres, en les intrégrant avec du HTML, du CSS et du JavaScript\footnote{Ce sont des languages de programmation pour le web, voir GLOSSAIRE.};
%     \item Le \textbf{pôle Web-Applicatif}, chargé de la réalisation de sites web e-commerce spécialiste dans les réseaux de transport et dans le tourisme. Les sites sont produits avec le framework\footnote{Ensemble d'outils, voir GLOSSAIRE.} \textsc{Symfony}\footnote{Framework PHP\cite{symfony}.} 
%     mais aussi avec les CMS\footnote{Logiciel en ligne pour créer un site web, voir GLOSSAIRE.} \wpress\~\footnote{Le CMS le plus utilisé\cite{wordpress}.} ou encore \textsc{Typo3}\footnote{Un autre CMS utilisé chez \nom\~\cite{typo3}.}.
% \end{itemize}
