\chapter[~~~PRÉSENTATION DE L'APPLICATION]{~~~I -~Présentation de l'application}%
\label{refDev1}%

Ce premier chapitre présente l'application \nom\~et son contexte. Il détaille le sujet du projet, les choix réalisés par rapport à ce dernier et les extensions effectuées.

\section{Résumé du sujet}
\subsection{Introduction}

Ce projet est un projet de fin de seconde année de licence d'informatique au sein de l'université de Franche-comté, dans l'unité d'enseignement Programmation Objet Avancée.
Le projet a été réalisé en binôme, avec Monsieur CUINET Antoine et Monsieur AMIOTTE-SUCHET Tristan. 

\subsection{Le projet}

Ce projet est réalisé en java (compatible java 11) en utilisant une programmation
objet mettant en œuvre les concepts d’héritage, de classe abstraite, d’interface, de généricité et de polymorphisme.
\bigskip

Ce projet est un exercice de programmation consistant à faire évoluer, automatiquement et au hasard, des personnages sur une grille où sont disposées des tours. 

Les personnages se déplacent d'une case à la fois en suivant une direction horizontale, verticale ou diagonale quand ils sont au sol. 
Ils rebondissent quand ils rencontrent un bord. 

Quand ils arrivent sur une case contenant une tour, ils entrent dedans se mettent à se déplacer en hauteur: ils peuvent monter les étages un à un jusqu'à arriver sur le toit. S'ils y parviennent ils deviennent propriétaires de la tour. 
Depuis le toit ils peuvent soit redescendre les étages un à un, soit sauter
sur le toit d'une autre tour à condition d'en être propriétaire.


\section{Choix par rapport au sujet}

Le sujet de base est un jeu de stratégie où les personnages évoluent de manière aléatoire sur une grille mais il est conseillé de le modifier afin de rendre les joueurs plus \g{intelligents}.

Nous avons donc implémenté une version où les personnages évoluent\dots, en respectant \textcolor{cardinal}{règles de fonctionnement à décrire}. 

\textcolor{cardinal}{Actutres types d'objets à placer sur la grille à faire}.


\section{Extensions réalisées}

\textcolor{cardinal}{Extensions réalisées}.

% \begin{itemize}
%     \item Le \textbf{pôle Référencement et Webmarketing}, qui a pour but de concevoir des stratégies de marketing, des campagnes influenceurs, etc et s'occupe du référencement naturel (SEO) et payant (SEA) (voir GLOSSAIRE); 
%     \item Le \textbf{pôle Web}, s'occupant de la création de sites web e-commerce dans le domaine de l'automobile, du sport et autres, en les intrégrant avec du HTML, du CSS et du JavaScript\footnote{Ce sont des languages de programmation pour le web, voir GLOSSAIRE.};
%     \item Le \textbf{pôle Web-Applicatif}, chargé de la réalisation de sites web e-commerce spécialiste dans les réseaux de transport et dans le tourisme. Les sites sont produits avec le framework\footnote{Ensemble d'outils, voir GLOSSAIRE.} \textsc{Symfony}\footnote{Framework PHP\cite{symfony}.} 
%     mais aussi avec les CMS\footnote{Logiciel en ligne pour créer un site web, voir GLOSSAIRE.} \wpress\~\footnote{Le CMS le plus utilisé\cite{wordpress}.} ou encore \textsc{Typo3}\footnote{Un autre CMS utilisé chez \nom\~\cite{typo3}.}.
% \end{itemize}
