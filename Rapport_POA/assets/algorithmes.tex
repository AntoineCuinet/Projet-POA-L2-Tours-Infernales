%%% algo html pour ginko
\begin{algorithm}
    \caption[\emph{Animation site de Ginko (code HTML)}]{\label{HTML_ginko}Animation site de Ginko (code HTML).}
    %\textbf{\textcolor{teal}{<!\,-\,- commentaire -\,->}}\\
    \textbf{<!\textcolor{blue}{DOCTYPE} \textcolor{cyan}{html}>}\\
    \html{}{
      \head{}{
      \smallskip
      
    \textbf{<\textcolor{blue}{title}>Animation pour le site de Ginko</\textcolor{blue}{title}>\\
    \textbf{\textcolor{teal}{<!\,-\,- Lien vers le fichier CSS -\,->}}\\
    <\textcolor{blue}{link} \textcolor{cyan}{rel}=\textcolor{auburn}{“stylesheet”} \textcolor{cyan}{href}=\textcolor{auburn}{“styles.css”}>\\}
    \smallskip
        
    }
    \smallskip
    
    \body{}{
    \smallskip
    
    \div{}{
    \textbf{<\textcolor{blue}{div} \textcolor{cyan}{class}=\textcolor{auburn}{text}>complet</\textcolor{blue}{div}>}\\
    \textbf{<\textcolor{blue}{div} \textcolor{cyan}{class}=\textcolor{auburn}{text}>2min</\textcolor{blue}{div}>}\\
    \textbf{<\textcolor{blue}{div} \textcolor{cyan}{class}=\textcolor{auburn}{text}>complet</\textcolor{blue}{div}>}\\
    }
    \smallskip
    
    }
}
\end{algorithm}
\bigskip


%%% algo css pour ginko
\begin{algorithm}
    \caption[\emph{Animation site de Ginko (code CSS)}]{\label{CSS_ginko}Animation site de Ginko (code CSS).}
    \point{\textbf{\textcolor{auburn}{container $ \{ $ }}}{
    height: \textcolor{blue}{15}px;\\
    margin: \textcolor{blue}{0};\\
    }
    \smallskip

    \point{\textbf{\textcolor{auburn}{text $ \{ $ }}}{
    text-align: \textcolor{cardinal}{center};\\
    font-size: \textcolor{blue}{15}px;\\
    color: \textcolor{cardinal}{black};\\
    \textcolor{teal}{/* animation (nom; durée; nombres de répétitions) */}\\
    animation: \textcolor{darkgray}{text-switch} \textcolor{blue}{8}s \textcolor{cardinal}{infinite};\\
    \textcolor{teal}{/* temps avant de commencer l'animation */}\\
    animation-delay: \textcolor{blue}{2}s;\\
    }
    \smallskip

    \keyf{\textbf{\textcolor{blue}{text-switch} \textcolor{auburn}{$ \{ $ }}}{
    \textcolor{teal}{/* 2 valeurs de \% à chaque fois afin de faire un arrêt sur le text */}\\ 
    \textcolor{darkgray}{0\%, 10\%} \{ transform: \textcolor{purple}{translateY}\,(\textcolor{blue}{0}); \}\\
    \textcolor{darkgray}{40\%, 60\%} \{ transform: \textcolor{purple}{translateY}\,(\textcolor{blue}{-100}\%); \}\\
    \textcolor{darkgray}{90\%, 100\%} \{ transform: \textcolor{purple}{translateY}\,(\textcolor{blue}{-200}\%); \}\\

    }
\end{algorithm}