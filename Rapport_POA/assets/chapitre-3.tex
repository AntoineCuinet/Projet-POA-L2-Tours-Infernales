\chapter[~~~DÉVELOPPEMENT]{~~~III -~Développement de l’application}%
\label{refDev3}%

Ce chapitre explique le développement de l'application \nom\~, il détaille les points intéressants du développement, le partage du travail et ce qui a été réalisé ou non.

\section{Illustration de points intéressants}

\textcolor{cardinal}{détailler des points intéressants}


\section{Partage du travail}

\textcolor{cardinal}{partage du travail}


\section{Ce qui a été réalisé}

Cette section détaille ce qui a été réalisé dans le projet \nom\~et ce qui a été testé.

\subsection{Ce qui a été développé}

\textcolor{cardinal}{ce qui a été développé}


\subsection{Ce qui a été testé}

\textcolor{cardinal}{ce qui a été testé}


\subsection{Ce qui n'a pas été implanté}

En soit, tout ce qui a été demandé dans le sujet a été réalisé. Cependant, nous n'avons pas eu le temps (avec les autres projets et les partiels) de rendre les personnages plus intelligents, c'est-à-dire de mettre en place des stratégies sous la forme de priorité à appliquer pour les déplacements des personnages. 

Nous avons donc laissé les personnages se déplacer de manière aléatoire.